%% start of file `template.tex'.
%% Copyright 2006-2013 Xavier Danaux (xdanaux@gmail.com).
%
% This work may be distributed and/or modified under the
% conditions of the LaTeX Project Public License version 1.3c,
% available at http://www.latex-project.org/lppl/.


\documentclass[11pt,a4paper,sans]{moderncv}        % possible options include font size ('10pt', '11pt' and '12pt'), paper size ('a4paper', 'letterpaper', 'a5paper', 'legalpaper', 'executivepaper' and 'landscape') and font family ('sans' and 'roman')

\usepackage{fontspec}
\setsansfont[
BoldFont=LinLibertine_RB.otf,
ItalicFont=LinLibertine_RI.otf,
BoldItalicFont=LinLibertine_RBI.otf
]{LinLibertine_R.otf}

% casual (default), classic, oldstyle or banking
\moderncvstyle{classic}
% blue (default), orange, green, red, purple, grey or black
\moderncvcolor{purple}
\usepackage{calc}
\usepackage[shortlabels,inline]{enumitem}
\usepackage[scale=0.75]{geometry}
\usepackage[defaultbib]{bibtopic}
\usepackage[unicode]{hyperref}

\newcommand{\textover}[3][l]{%
  % #1 is the alignment, default l
  % #2 is the text to be printed
  % #3 is the text for setting the width
  \makebox[\widthof{#3}][#1]{#2}%
}

\hypersetup{
    colorlinks=true,
    linkcolor=color1,
    citecolor=color1,
    filecolor=color1,
    urlcolor=color1
}

\name{Hugo}{Mougard}

\address{18 rue Paul Ramadier}%
{44200 Nantes, France}

\phone[mobile]{+33~(0)~6~37~63~82~71}

\email{hugo@mougard.fr}

\social[github]{m09}

\photo[64pt]{photo}

\begin{document}

\makecvtitle

\section{Experience}

\cventry{08/18–\textover{now}{03/18}}%
{Senior Machine Learning Engineer}%
{source\{d\}}%
{Nantes, France}%
{}%
{Empower developers by producing data-aware tooling.}

\cventry{03/17–{08/18}}%
{Machine Learning Consultant}%
{Freelance}%
{Nantes, France}%
{}%
{ML, NLP, NLU and CV consultancy \& training.}

\cventry{10/16–02/17}%
{Deep Learning Engineer}%
{CogniTalk}%
{Nantes, France}%
{}%
{NLP and NLU engineering.}

\cventry{09/14–09/16}%
{Ph.D. Student}%
{University of Nantes}%
{Nantes, France}%
{}%
{Research multimodal content alignment using deep learning (CV, NLP
  and NLU research).}

\cventry{03/14–08/14}%
{Research Intern}%
{National Institute of Informatics}%
{Tokyo, Japan}%
{}%
{Simplify text automatically (NLP research).}

\cventry{04/13–07/13}%
{Research Intern}%
{Dublin City University}%
{Dublin, Republic of Ireland}%
{}%
{Retrieve information in medical records (NLP research).}

\cventry{04/12–07/12}%
{Research Intern}%
{University of Nantes}%
{Nantes, France}%
{}%
{Study and organize grammatical inference litterature and code
  (NLP research).}

\section{Education}

\cventry{2014–2016}%
{Ph.D. in Computer Science}%
{University of Nantes}%
{Nantes, France}%
{\textbf{undefended}}%
{Research ways to align multimodal content with deep learning.}

\cventry{2012–2014}%
{M.Sc. in Computer Science}%
{University of Nantes}%
{Nantes, France}%
{}%
{Specialization in Machine Learning \& Natural Language Processing}

\cventry{2009–2012}%
{B.Sc. in Computer Science}%
{University of Nantes}%
{Nantes, France}%
{}%
{}

\section{Research Interests}
\cvitem{}{My research interests are centered around natural language
  processing, natural language understanding and machine learning on
  code.}

\section{Software}
\cvitem{Frameworks}{Good knowledge and experience in several machine
  learning and natural language processing frameworks: PyTorch, DGL,
  sklearn, bigARTM, Stanford CoreNLP, Gensim, UIMA (non-exhautive)}

\cvitem{Programming languages}{Language agnostic with more experience
  in Python, Java and Bash.}

\section{Publications}

\bibliographystyle{unsrt}

\subsection{International Conferences}
\begin{btSect}{confs}
\btPrintNotCited
\end{btSect}

\subsection{International Workshops}
\begin{btSect}{workshops}
\btPrintNotCited
\end{btSect}

\subsection{Technical Reports}
\begin{btSect}{technical-reports}
\btPrintNotCited
\end{btSect}

\section{Workshops}

\begin{itemize}
\item \href{https://github.com/m09/deeplearning-codelab}{Understand
    your code with Machine Learning on Source Code}, DevFest Nantes,
  2019.
\item \href{https://github.com/m09/deeplearning-codelab}{The Deep
    Learning Codelab} (in French), DevFest Nantes, 2017.
\item
  \href{https://github.com/nantes-machine-learning-meetup/NMLM/tree/master/2015-10-05__r\%C3\%A9gression-lin\%C3\%A9aire-logistique}{
    Linear \& Logistic Regression} (in French), Nantes Machine
  Learning Meetup, 2015.
\end{itemize}

\section{Talks}

\subsection{2019}

\begin{itemize}
\item
  \href{https://www.meetup.com/Nantes-Machine-Learning-Meetup/events/265265431/}{Overton,
    Apple-flavored ML} (in French), Nantes Machine Learning Meetup.
\item
  \href{https://www.eventbrite.com/e/tech-environmental-collapse-tickets-57986002695}{Tech
    \& Environmental Collapse}, source\{d\} Meetup.
\end{itemize}

\subsection{2018}

\begin{itemize}
\item
  \href{https://www.meetup.com/GOTO-Nights-CPH/events/256342503/}{Bootstrapping
    Machine Learning on Code with ideas from NLP}, GOTO Nights
  Copenhagen.
\item
  \href{https://www.mapado.com/nantes/la-matinale-de-la-data-science-et-du-machine-learning}{Ins
    and outs of Machine Learning} (in French), Data Science and
  Machine Learning Special, ENI.
\end{itemize}
\subsection{2017}

\begin{itemize}
\item
  \href{https://www.meetup.com/Nantes-Machine-Learning-Meetup/events/239481485/}{Generative
    Adversarial Networks, a New Paradigm to Train ANNs} (in French),
  Nantes Machine Learning Meetup.
\end{itemize}

\subsection{2016}

\begin{itemize}
\item \href{https://youtu.be/xv2S8A1EPqI}{Artificial Intelligence: How
    Google Defeated One of the Best Go Players} (in French), Nantes
  University Public Conference.
\item \href{https://youtu.be/KuvXb2nILLc}{Solving Go with Machine
    Learning} (in French), PhD Society (LOGIN) Talk.
\item
  \href{https://www.meetup.com/Nantes-Machine-Learning-Meetup/events/226648150/}{Neural
    Programming} (in French), Nantes Machine Learning Meetup.
\end{itemize}

\subsection{2015}

\begin{itemize}
\item
  \href{https://www.meetup.com/Nantes-Machine-Learning-Meetup/events/221108033/}{Sequence
    to Sequence Learning with Neural Networks} (in French), Nantes
  Machine Learning Meetup.
\end{itemize}

\section{Languages}
\cvitem{French}{\textbf{Mother tongue}}{}
\cvitem{English}{\textbf{Expert}}{}
\cvitem{Spanish}{Good reading comprehension}{}
\cvitem{Japanese}{Basic knowledge}{}

\end{document}


% %% end of file `template.tex'.
%%% Local Variables: 
%%% coding: utf-8
%%% mode: latex
