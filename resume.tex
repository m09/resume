%% start of file `template.tex'.
%% Copyright 2006-2013 Xavier Danaux (xdanaux@gmail.com).
%
% This work may be distributed and/or modified under the
% conditions of the LaTeX Project Public License version 1.3c,
% available at http://www.latex-project.org/lppl/.


\documentclass[11pt,a4paper,sans]{moderncv}        % possible options include font size ('10pt', '11pt' and '12pt'), paper size ('a4paper', 'letterpaper', 'a5paper', 'legalpaper', 'executivepaper' and 'landscape') and font family ('sans' and 'roman')

\usepackage{fontspec}

\setsansfont{Linux Libertine O}

% casual (default), classic, oldstyle or banking
\moderncvstyle{classic}
% blue (default), orange, green, red, purple, grey or black
\moderncvcolor{purple}

\usepackage[shortlabels,inline]{enumitem}
\usepackage[scale=0.75]{geometry}

\usepackage[unicode]{hyperref}
\hypersetup{
    colorlinks=true,
    linkcolor=color1,
    citecolor=color1,
    filecolor=color1,
    urlcolor=color1
}

\name{Hugo}{Mougard}

\address{LINA, Université de Nantes}%
{2 rue de la Houssinière}%
{44322 Nantes Cedex 03, France}

\phone[mobile]{+33~(0)~6~37~63~82~71}

\phone[fixed]{+33~(0)~2~51~12~53~29}

\email{hugo.mougard@univ-nantes.fr}

\homepage{http://crydee.eu}

\social[github]{m09}

\photo[64pt]{photo}

\begin{document}

\makecvtitle

\section{Education}
\cventry{2014–current}{Ph.D. in Computer Science}{University of
  Nantes}{}{}{Subject: Multimodal Alignment for Educational Resources}
\cventry{2012–2014}{M.Sc. in Computer Science}{University of
  Nantes}{}{}{\href{http://atal.univ-nantes.fr}{ATAL} specialization
  in Machine Learning \& Natural Language Processing}
\cventry{2009–2012}{B.Sc. in Computer Science}{University of Nantes}{}{}{}

\section{Experience}
\subsection{Internships}
\cventry{2014}{Text simplification}{National Institude of
  Informatics}{Tokyo}{5 months}{%
  Contribute to the Gaze-N.L.P. team work in Aizawa Laboratory:
  \begin{itemize}
  \item study literature and find a research question: “How to best
    learn to simplify text using Simple Wikipedia edit history as a
    weak learning signal?” (1 month);
  \item develop of a corpus and related tools (2 months),
    \href{https://github.com/m09/readability}{publicly available};
  \item investigate simplification techniques (1 month);
  \item write-up the internship report (2 weeks).
  \end{itemize}
}

\cventry{2013}{Information Retrieval}{Dublin City
  University}{Dublin}{3 months}{%
  Contribute to the Information Retrieval team work in several
  aspects:
  \begin{itemize}
  \item study the field through the available literature;
  \item refactor, expand and evaluate an existing system for request
    expansion;
  \item help the SIGIR '2013 team as a volounteer.
  \end{itemize}
}

\cventry{2012}{Grammatical Inference}{Laboratoire d'Informatique
  Nantes Atlantique}{Nantes}{2 months}{%
  Gather, organize and make accessible the state of the art algorithms
  implementations in Grammatical Inference:
  \begin{itemize}
  \item discover the field through a bibliographic study;
  \item spot and acquire implementations of the main algorithms by
    contacting the main researchers in the field and the winners of
    the competitions organized on the matter;
  \item publish the resulting software on a website.
  \end{itemize}
}

\nocite{*}
\bibliographystyle{apalike}
\begin{thebibliography}{}

\bibitem[Mougard et~al., 2015]{mougard15:_paper_video}
Mougard, H., Riou, M., de~la Higuera, C., Quiniou, S., and Aubert, O. (2015).
\newblock The paper of the video: Why choose?
\newblock In {\em Proceedings of the SAVE-SD Workshop of the 2015 World Wide
  Web Conference}, page In press.

\end{thebibliography}

\section{Research Interests}
\cvitem{}{My research interests are alignment algorithms, multimodal
  Information Retrieval, Machine Learning applied to Natural Language
  Processing and educative and scholarly data use.}

\section{Skills}
\cvitem{Software}{Extensive knowledge of \textbf{UIMA}, very good
  knowledge of Weka, OpenNLP and Stanford CoreNLP, good knowledge of
  Lucene, GIZA++ and other central NLP frameworks and libraries.}

\cvitem{Programming languages}{\textbf{Language agnostic}, comfortable
  in Java, Python, Bash, Scala, Prolog, Haskell, C, etc.}

\section{Languages}
\cvitem{French}{\textbf{Mother tongue}}{}
\cvitem{English}{\textbf{Expert}}{}
\cvitem{Spanish}{Good reading comprehension}{}
\cvitem{Japanese}{Basic knowledge}{}

\section{References}
\begin{itemize}
\item Dr.~Colin de la Higuera, Professor at Nantes University,
  <\href{mailto:cdlh@univ-nantes.fr}{cdlh@univ-nantes.fr}>
\item Dr.~Solen Quiniou, Assistant Professor at Nantes University,
  <\href{mailto:solen.quiniou@univ-nantes.fr}{solen.quiniou@univ-nantes.fr}>
\item Dr.~Akiko Aizawa, Professor at NII and University of Tokyo,
  <\href{mailto:aizawa@nii.ac.jp}{aizawa@nii.ac.jp}>
\item Dr.~Lorraine Goeuriot, Assistant Professor at Université Joseph
  Fourier,
  <\href{mailto:lorraine.goeuriot@imag.fr}{lorraine.goeuriot@imag.fr}>
\item Dr.~Florian Boudin, Assistant Professor at Nantes University,
  <\href{mailto:florian.boudin@univ-nantes.fr}{florian.boudin@univ-nantes.fr}>
\end{itemize}

% Publications from a BibTeX file without multibib
%  for numerical labels: \renewcommand{\bibliographyitemlabel}{\@biblabel{\arabic{enumiv}}}% CONSIDER MERGING WITH PREAMBLE PART
%  to redefine the heading string ("Publications"): \renewcommand{\refname}{Articles}

% \clearpage
% %-----       letter       ---------------------------------------------------------
% % recipient data
% \recipient{LxMLS 2015}{Instituto Superior Técnico\\%
% Lisbon, Portugal}
% \date{April 8, 2015}
% \opening{To whom it may concern,}
% \closing{Best regards,}
% \enclosure[Attached]{curriculum vit\ae{}}          % use an optional argument to use a string other than "Enclosure", or redefine \enclname
% \makelettertitle

% The reason why we want to attend LxMLS is that our team is focused on
% Machine Learning applied to NLP in different domains (Information
% Retrieval, Discourse Analysis, Multimodal Alignment, etc). Therefore
% the high quality courses and labs of the Lisbon Machine Learning
% Summer School would allow us to explore state of the art Deep Learning
% techniques in several domains, which is of utmost interest to us.

% Personnally, I started a Ph.D. 6 months ago that is deeply linked to
% paraphrase detection, topic modeling and several other language
% comprehension tasks to be able to correctly align multimodal
% documents. The power of the deep learning models in those tasks has
% already been demonstrated and I can't wait to learn more about them in
% the LxMLS.

% Furthermore, this summer school is for me an unique opportunity to
% meet like-minded students and great experts (professionnals and
% researchers). I wish and hope that this will lead to interesting
% collaboration opportunities.

% \makeletterclosing

% %\clearpage\end{CJK*}                              % if you are typesetting your resume in Chinese using CJK; the \clearpage is required for fancyhdr to work correctly with CJK, though it kills the page numbering by making \lastpage undefined
\end{document}


% %% end of file `template.tex'.
