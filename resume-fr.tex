\documentclass[11pt,a4paper]{moderncv}

\usepackage{fontspec}
\usepackage{calc}
\usepackage[shortlabels,inline]{enumitem}
\usepackage[scale=0.75]{geometry}
\usepackage[defaultbib]{bibtopic}
\usepackage[unicode]{hyperref}

\setmainfont{Linux Libertine O}
\moderncvstyle{casual}
\moderncvcolor{purple}
\bibliographystyle{unsrt}

\hypersetup{%
    colorlinks=true,%
    linkcolor=color1,%
    citecolor=color1,%
    filecolor=color1,%
    urlcolor=color1%
}

\name{Hugo}{Mougard}
\address{55 rue du Coudray}{44000 Nantes, France}
\phone[mobile]{+33~(0)~6~37~63~82~71}
\email{hugo@mougard.fr}
\social[github]{m09}
\photo[64pt]{photo}

\begin{document}

\makecvtitle

\section{Expérience}

\cventry{01/2020–courant}%
{Consultant et formateur en Machine Learning}%
{Freelance}%
{Nantes, France}%
{}%
{Consulting et formation en ML, NLP, NLU, CV et Python.}

\cventry{08/18–01/20}%
{Ingénieur Machine Learning senior}%
{source\{d\}}%
{Nantes, France}%
{}%
{Conception d'outillage de développement logiciel en utilisant le machine learning.}

\cventry{03/17–08/18}%
{Consultant et formateur en Machine Learning}%
{Freelance}%
{Nantes, France}%
{}%
{Consulting et formation en ML, NLP, NLU, et CV.}

\cventry{10/16–02/17}%
{Ingénieur Deep Learning}%
{CogniTalk}%
{Nantes, France}%
{}%
{Création de chatbots en utilisant le deep learning (NLP et NLU).}

\cventry{09/14–09/16}%
{Étudiant en thèse}%
{Université de Nantes}%
{Nantes, France}%
{}%
{Recherche sur l'alignement de contenus multi-modaux utilisant le deep learning (CV, NLP et NLU).}

\cventry{03/14–08/14}%
{Stage de recherche de M2}%
{National Institute of Informatics}%
{Tokyo, Japon}%
{}%
{Simplification automatique de texte (recherche en NLP).}

\cventry{04/13–07/13}%
{Stage de recherche de M1}%
{Dublin City University}%
{Dublin, République d'Irelande}%
{}%
{Recherche d'information dans des enregistrements médicaux (recherche en NLP).}

\cventry{04/12–07/12}%
{Stage de recherche de L3}%
{Université de Nantes}%
{Nantes, France}%
{}%
{Étude bibliographie de l'ingérence grammaticale (recherche en NLP).}

\section{Éducation}

\cventry{2014–2016}%
{Thèse en informatique}%
{Université de Nantes}%
{Nantes, France}%
{\textbf{non soutenue}}%
{Recherche sur l'alignement de contenus multi-modaux utilisant le deep learning (CV, NLP et NLU).}

\cventry{2012–2014}%
{Master en informatique}%
{Université de Nantes}%
{Nantes, France}%
{}%
{Spécialisation en machine learning \& natural language processing}

\cventry{2009–2012}%
{Licence en informatique}%
{Université de Nantes}%
{Nantes, France}%
{}%
{}

\section{Recherche}
\cvitem{}{Les sujets de recherche qui me passionnent sont centrés autour du traitement
  du langage naturel, de la compréhension du langage et du machine learning sur du code
  source.}

\section{Software}
\cvitem{Frameworks}{Expérience et bonne connaissance de plusieurs frameworks de machine
  learning~: PyTorch, DGL, sklearn, bigARTM, Stanford CoreNLP, Gensim, UIMA
  (non-exhautif)}

\cvitem{Langages de programmation}{J'écris majoritairement en Python, Java et Bash/ZSH.}

\section{Publications}

\subsection{Conférences internationales}
\begin{btSect}{confs}
\btPrintNotCited
\end{btSect}

\subsection{Ateliers internationaux}
\begin{btSect}{workshops}
\btPrintNotCited
\end{btSect}

\subsection{Rapports techniques}
\begin{btSect}{technical-reports}
\btPrintNotCited
\end{btSect}

\section{Ateliers}

\begin{itemize}
\item \href{https://github.com/m09/deeplearning-codelab}{Understand
    your code with Machine Learning on Source Code}, DevFest Nantes,
  2019.
\item \href{https://github.com/m09/deeplearning-codelab}{The Deep
    Learning Codelab}, DevFest Nantes, 2017.
\item
  \href{https://github.com/nantes-machine-learning-meetup/NMLM/tree/master/2015-10-05__r\%C3\%A9gression-lin\%C3\%A9aire-logistique}{
    Linear \& Logistic Regression} (French), Nantes Machine
  Learning Meetup, 2015.
\end{itemize}

\section{Talks}

\subsection{2020}
\begin{itemize}
  \item
  \href{https://www.meetup.com/Nantes-Machine-Learning-Meetup/events/268243136/}{Faisons connaissance avec les Transformeurs}, Nantes Machine Learning Meetup.
\end{itemize}

\subsection{2019}

\begin{itemize}
\item
  \href{https://www.meetup.com/Nantes-Machine-Learning-Meetup/events/265265431/}{Overton, le ML goût pomme}, Nantes Machine Learning Meetup.
\item
  \href{https://www.eventbrite.com/e/tech-environmental-collapse-tickets-57986002695}{Tech
    \& Environmental Collapse}, source\{d\} Meetup.
\item \href{https://github.com/m09/talks/tree/master/kth}{Machine
    Learning for Large Scale Code Analysis}, séminaire KTH TCS.
\end{itemize}

\subsection{2018}

\begin{itemize}
\item
  \href{https://www.meetup.com/GOTO-Nights-CPH/events/256342503/}{Bootstrapping
    Machine Learning on Code with ideas from NLP}, GOTO Nights
  Copenhagen.
\item
  \href{https://www.mapado.com/nantes/la-matinale-de-la-data-science-et-du-machine-learning}{Ins
    and outs of Machine Learning}, spéciale data science \& machine learning, ENI.
\end{itemize}

\subsection{2017}

\begin{itemize}
\item
  \href{https://www.meetup.com/Nantes-Machine-Learning-Meetup/events/239481485/}%
  {Generative Adversarial Networks, un nouveau paradigme pour l'entraînement d'ANN},
  Nantes Machine Learning Meetup.
\end{itemize}

\subsection{2016}

\begin{itemize}
\item \href{https://youtu.be/xv2S8A1EPqI}{Intelligence artificielle : comment Google a
  battu l'un des meilleurs joueurs de Go}, Conférence publique de l'Université de Nantes.
\item \href{https://youtu.be/KuvXb2nILLc}{Solving Go with Machine
    Learning}, demi-heure du doctorant de l'association LOGIN.
\item
  \href{https://www.meetup.com/Nantes-Machine-Learning-Meetup/events/226648150/}{Neural
    Programming}, Nantes Machine Learning Meetup.
\end{itemize}

\subsection{2015}

\begin{itemize}
\item
  \href{https://www.meetup.com/Nantes-Machine-Learning-Meetup/events/221108033/}{Sequence
    to Sequence Learning with Neural Networks}, Nantes
  Machine Learning Meetup.
\end{itemize}

\section{Langues}
\cvitem{Français}{\textbf{Langue maternelle}}{}
\cvitem{Anglais}{\textbf{Expert}}{}
\cvitem{Espagnol}{Bonne compréhension écrite}{}
\cvitem{Japonais}{Niveau N4 / A2}{}

\end{document}
